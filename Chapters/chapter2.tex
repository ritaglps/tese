%!TEX root = ../template.tex
%%%%%%%%%%%%%%%%%%%%%%%%%%%%%%%%%%%%%%%%%%%%%%%%%%%%%%%%%%%%%%%%%%%%
%% chapter2.tex
%% NOVA thesis document file
%%
%% Chapter with the template manual
%%%%%%%%%%%%%%%%%%%%%%%%%%%%%%%%%%%%%%%%%%%%%%%%%%%%%%%%%%%%%%%%%%%%

\typeout{NT FILE chapter2.tex}%

\chapter{Estado da Arte}
\label{cha:state_of_the_art}

\glsresetall

\section{Introduction}
\label{sec:introductionn}

This Chapter describes how to use the \gls{novathesis}\ template.  It is assumed that you have a working \index{installation} of \LaTeX, either local (in your own computer) or remote, and that you were able to generate a PDF for the default configuration of the template: a PhD thesis for \gls{FCT}.

\section{Example glossary, acronyms, and symbols}
Be carefull with mathematical symbols in acronyms, please see the definition of \gls{mu}.

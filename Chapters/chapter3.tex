%!TEX root = ../template.tex
%%%%%%%%%%%%%%%%%%%%%%%%%%%%%%%%%%%%%%%%%%%%%%%%%%%%%%%%%%%%%%%%%%%%
%% chapter3.tex
%% NOVA thesis document file
%%
%% Chapter with a short latex tutorial and examples
%%%%%%%%%%%%%%%%%%%%%%%%%%%%%%%%%%%%%%%%%%%%%%%%%%%%%%%%%%%%%%%%%%%%

\typeout{NT FILE chapter3.tex}%

\makeatletter
\newcommand{\ntifpkgloaded}{%
  \@ifpackageloaded%
}
\makeatother


\chapter{Tecnologias}
\label{cha:technologies}

\section{Document Structure} % (fold)
\label{sec:document_structure}

\section{Dealing with Bibliography} % (fold)
\label{sec:dealing_with_bibliography}

Citing something online~\cite{wiki:shuntingyard,flex,bison}.

\section{Floats, Figures and Captions} % (fold)
\label{sec:floats_figures_and_captions}

\begin{figure}[htbp]
  \centering
  \subbottom[One sub-figure\label{fig:leftsubfig}]{%
    \includegraphics[width=0.5\linewidth]{knitting-vectorial}}%
  \subbottom[Another sub-figure\label{fig:rightsubfig}]{%
    \includegraphics[width=0.5\linewidth]{knitting-vectorial}}%
  \caption{A figure with two sub-figures!}
  \label{fig:fig2subfig}
\end{figure}

\textbf{And this is a small text that references the Figure~\ref{fig:fig2subfig} and its Subfigures~\ref{fig:leftsubfig} and~\ref{fig:rightsubfig}.}

\subsection{Footnotes} % (fold)

Footnotes\footnote{This is a simple footnote.} will be numbered and shown in the bottom of the page.

\subsection{Tables} % (fold)

\bgroup
\rowcolors{1}{}{GhostWhite}
\begin{xltabular}{\textwidth}{Xccccc}
  \caption{Test results summary.}
  \label{tab:hla:results}\\
  \toprule
  \rowcolor{Gainsboro}%
  Test   & Anomalies  & Warnings  & Correct   & Categories            & Missed \\
  \midrule
Connection~\cite{Beckman08}     & 2       & 2          & 1          & \emph{C}              & 1 \\
Coordinates'03~\cite{Artho03}   & 1        & 4          & 1          & \emph{2B, 1C}          & 0 \\
Local Variable~\cite{Artho03}    & 1        & 2          & 1          & \emph{A}              & 0 \\
NASA~\cite{Artho03}              & 1        & 1          & 1          & ---                    & 0 \\
Knight Moves~\cite{Beckman08}   & 1        & 3          & 1          & \emph{2B}              & 0 \\
  \midrule
  \rowcolor{Gainsboro}%
Total                            & 12      & 33        & 10        & 5A, 6B, 10C, 2D       & 2 \\
  \bottomrule
  \end{xltabular}
\egroup
